\section{DISCUSSION}\label{s:discussion}

The fascinating conclusion from this study is that the experimental results show the opposite effect to the initial hypothesis. The initial hypothesis states that the psychological salience of red as a color related to danger and aggressiveness would prompt subjects to overestimate the presence of red players. The experimental results show that subjects overestimated the presence of blue players. We provide a few speculative explanations for this behavior below.

\subsection{Results Explanations}

One possible explanation is a failure of the assumption that higher salience means higher numerical estimates. It could be that red still has a higher salience, which leads to more accurate counts of red players, and that the lower salience of blue actually leads subjects to overestimate the number of blue players due to the inability to accurately count them. This would indicate a bias towards overestimating objects that receive low cognitive attention, a plausible, conservative, survival mechanism. This theory could be tested with a similar experiment where the subjects are tasked to estimate the counts of salient and non-salient objects, perhaps by manipulating the contrast on a fixed background. If the counts of non-salient objects are consistently overestimated it would explain the result in this study, while upholding the assumption that red is the more salient color. One confounding effect to consider in this proposed experiment is the natural bias of each subject, some subjects might overestimate when faced with uncertainty, while others might underestimate. This effect could be studied by providing an intuition-based numerical guessing task for each subject and tracking the effect of uncertainty in numerical guessing.

Another possible explanation is that blue might have a higher natural salience of its own. This salience can be biological, related to ocular receptors and wavelengths, or psychological. Red's psychological salience is well-researched and is often exploited in modern visual signals. Blue also has a likely psychological salience of its own kind, it is a color rarely found in land organisms, but often associated with water and the sky, two features that humans might intuitively search for. Blue's salience is also used in road signs, particularly in the European Union, which is a testament to its ability to draw attention. Perhaps this alternative psychological salience counteracts the salience of red and leads to a bias towards blue estimates. This theory could be tested by performing a study with more diverse colors, always including red and blue, and investigating where blue and red stand in a ranking of salience, if both blue and red place similarly in salience, perhaps this study is affected by two highly salient colors.

A final explanation to consider is some confounding effects due to the experiment design. One possible point of conflict is the labeling of buttons for "red" and "blue" with the color of the respective option. This is a visual guide to avoid errors in the button clicking. Perhaps red's association with danger and aggressiveness leads subjects to be more hesitant towards choosing the red option, creating a bias towards the more agreeable color, blue. This confounding effect can be studied with an additional experiment testing bias between red and blue labeled buttons, or by repeating this study without coloring the buttons.

Another interesting finding is that the effect size is stronger for females. This could point to a psychological effect imbued by societal norms or could be a factor of the different sample sizes of males vs females. This finding should also be studied further, with studies that focus on the differences in biases and attention to blue and red, based on sex and gender. Such a study could determine whether there is a biological or a socio-psychological effect on color-based attention. The results also show that the effect is larger for subjects who have never played football. Given football in Europe has traditionally been more popular among males, this might explain why the effect is larger in females. This study could be repeated with a focus on tracking familiarity with football as an independent variable, to check if it is correlated to gender and the effect on player detection.


\subsection{Applicability to the real world}
The findings in this paper could explain the red-advantage effect in football \cite{english}. The amount of players wearing red is often underestimated, this adds a component of stealth that can be exploited by players. In fact, the original assumption that red's psychological salience leads to the overestimation of player counts, makes the red-advantage effect harder to justify. Overestimating opposing red players could lead to more conservative play and stop potentially good plays, but underestimating red players is likely to lead to mistakes and lost balls which are often more detrimental. A further study should be done to verify which of these effects is more beneficial to team performance, being numerically overestimated or numerically underestimated. If such an effect is significant, clothing colors should be increasingly taken into account to improve competitiveness and fairness.

Another interesting use of this finding is in signaling techniques, where we can exploit the presence of blue being overestimated.  Any situation that uses red to signal several dangers or measurements could be switched to blue if a less accurate and more conservative perception is desirable. For instance, red lights are often used in aviation to signal obstacles, using blue lights might lead to a more conservative estimation of the danger of an area, leading to safer flying. Another example is the bars with red and white stripes that are often used as barriers for cars. Perhaps using blue and white stripes will lead to the bar being perceived as longer, avoiding collisions.

