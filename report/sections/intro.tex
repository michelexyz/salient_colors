\section{Introduction}\label{s:intro}

Color psychology is prevalent in our daily lives, and it is also evident in sports. Prior research shows that the choice of team colors in sports, beyond aesthetic considerations, plays a meaningful role in influencing the psyche of athletes, spectators, and officials. Our objective is to investigate whether or not the color of a team's jersey impacts the salience of the players, this could imply a competitive advantage where teammates detect each other more easily. 

Through an experiment involving displaying virtual match screenshots to users, the objective is to attain statistical measurements of the critical role colors play in molding the psychological dynamics within the realm of salience in sports. Accomplishing this objective aims to provide insights that illuminate the subtleties inherent in the field of color psychology as it pertains to athletics. 

\subsection{Background}\label{s:background}

Red and blue are two colors that have been widely investigated in psychological research, and they also have been analyzed in sports. Red often signifies dominance and aggression, whereas blue conveys calmness and trust, according to psychological studies \cite{frontiers}. These associations may wield psychological influence on players, referees, and spectators alike. In this study, red and blue uniforms are selected as a means to investigate the psychological effects of these colors on salience within the context of football. 

The color of football outfits on the visual perception and location assessment of football players, or salience, has also been discussed. The colors are measured with the HSB (hue, saturation, brightness) scale to analyze the visibility of players in regards to the brightness and saturation of a team jersey color on the salience of a player \cite{colors}. First, the colors were chosen based on contrast to the background, a virtual green field, in which white was concluded to be the most visible color using HSB analysis. The performance of the football teams with high visibility, or white jersey colors, was hypothesized to be better than those of less contrasted colors. The experiment concluded that, within a virtual environment, the positions of players wearing white uniforms were significantly better assessed than that of players wearing green \cite{colors}.

In other studies, football teams wearing red have been studied and analyzed to win more than teams wearing other colors, in particular blue teams. Multiple experiments and statistical analyses have concluded that red jerseys lead to successful matches in sports. A study of matched-pairs analysis of red and non-red-wearing teams for English football teams concluded that teams that wear red perform significantly better \cite{red}. The Wilcoxon signed rank test proved that teams with red jerseys won substantially more often than others. The authors concluded that the ‘‘red advantage’’ comes from visibility differences and psychological responses. Teams wearing red are perceived as more attractive to paying supporters due to the psychological association between success and red \cite{red}. 

The primary objective of this research is to use an experimental approach to obtain statistical measurements of the role of colors, specifically red or blue, in the psychological dynamics in football. The influence of color extends beyond the confines of the playing field, permeating into fan loyalty, team branding, and the overall spectator experience. Our experiment will measure whether or not red or blue jersey colors significantly affect player detection and salience on the football field.



% \subsection{Premier League: Effects of shirt color, team ability and time trends}

% In the past two decades, home advantage in sports has gained significant attention \cite{Home}. It involves a higher success rate for home teams compared to away teams and has been consistently observed in both individual and team sports \cite{Tennis}. Meta-analysis by Jamieson (2010) revealed a 60.4\% home winning rate across 87 samples and over 260,000 games. Courneya and Carron's (1992) feed-forward model is a well-established explanation for home advantage, linking game location factors like crowd influence, travel effects, location familiarity, and competition rules to the psychological states of competitors, coaches, and officials.

% Research has explored the impact of these factors, with crowd size and noise influencing referee decisions in favor of the home team and performance deteriorating with greater travel distance \cite{crowd}. Three factors (crowd, travel, and familiarity) have shown support for contributing to home advantage in team sports, while competition rules have received less attention.

% The color of team shirts, particularly red, has gained interest due to its psychological impact. Studies suggest that teams wearing red tend to have a higher chance of winning and elicit certain psychological responses from opponents and viewers \cite{english}. Teams choosing red shirts may benefit from a higher home winning percentage.

\subsection{Main Hypothesis}\label{s:overview}
 
The experiment aims to investigate the impact of red and blue uniforms on player salience in the context of sports. Red and blue are two colors with well-established associations in psychological research, where red is often linked to dominance and aggression, while blue conveys calmness and trust. These color associations may influence players, referees, and spectators in the world of sports. 

The main hypothesis of this study is that these subconscious associations can lead to different rates of detection between the two colors.

To conduct this study, we will select red and blue as the primary uniform colors for two football teams. These colors were chosen due to their contrasting psychological associations; and their similar contrast with the field, based on HSL values, so that that this wouldn't confound the effect of the first. We will assess how the jersey colors impact player salience against the backdrop of a virtual green field. Previous research has already suggested that colors with higher visibility may lead to better position assessment, for this reason, we opted to remove this variable to concentrate solely on the effect of cognitive associations with colors \cite{colors}.